%%%%%%%%%%%%%%%%%%%%%%%%%%%%%%%%%%%%%%%%%
% "ModernCV" CV and Cover Letter
% LaTeX Template
% Version 1.3 (29/10/16)
%
% This template has been downloaded from:
% http://www.LaTeXTemplates.com
%
% Original author:
% Xavier Danaux (xdanaux@gmail.com) with modifications by:
% Vel (vel@latextemplates.com)
%
% License:
% CC BY-NC-SA 3.0 (http://creativecommons.org/licenses/by-nc-sa/3.0/)
%
% Important note:
% This template requires the moderncv.cls and .sty files to be in the same 
% directory as this .tex file. These files provide the resume style and themes 
% used for structuring the document.
%
%%%%%%%%%%%%%%%%%%%%%%%%%%%%%%%%%%%%%%%%%

%----------------------------------------------------------------------------------------
%	PACKAGES AND OTHER DOCUMENT CONFIGURATIONS
%----------------------------------------------------------------------------------------

\documentclass[11pt,a4paper,sans]{moderncv} % Font sizes: 10, 11, or 12; paper sizes: a4paper, letterpaper, a5paper, legalpaper, executivepaper or landscape; font families: sans or roman

\moderncvstyle{banking} % CV theme - options include: 'casual' (default), 'classic', 'oldstyle' and 'banking'
\moderncvcolor{black} % CV color - options include: 'blue' (default), 'orange', 'green', 'red', 'purple', 'grey' and 'black'

\usepackage{lipsum} % Used for inserting dummy 'Lorem ipsum' text into the template
\usepackage[scale=0.75, margin=1in]{geometry} % Reduce document margins
\usepackage[inline]{enumitem} % Listing courses
\usepackage{amsmath}
\usepackage{tabularx} % in the preamble
\usepackage{footmisc}
\usepackage{multibib}
\usepackage{url}

\newcites{article}{Refereed Publications}
\newcites{journal}{Journals}
\newcites{proceeding}{Conference Papers}
\newcites{abstract}{Abstracts}

\newcites{dev}{Under Development}
%\newcites{papers}{Journals and Full Papers}
%\newcites{shorts}{Abstracts and Workshop Papers}
%\setlength{\hintscolumnwidth}{3cm} % Uncomment to change the width of the dates column
%\setlength{\makecvtitlenamewidth}{10cm} % For the 'classic' style, uncomment to adjust the width of the space allocated to your name

%----------------------------------------------------------------------------------------
%	NAME AND CONTACT INFORMATION SECTION
%----------------------------------------------------------------------------------------

\firstname{Matthew} % Your first name
\familyname{Piekenbrock} % Your last name

% All information in this block is optional, comment out any lines you don't need
\title{\\ Curriculum Vitae}
%\address{123 Broadway}{City, State 12345}
\mobile{(937) 269 8582}
%\phone{(000) 111 1112}
%\fax{(000) 111 1113}
\email{matt.piekenbrock@gmail.com}
\homepage{wright.edu/~piekenbrock.5}{wright.edu/$\sim$piekenbrock.5} % The first argument is the url for the clickable link, the second argument is the url displayed in the template - this allows special characters to be displayed such as the tilde in this example
% \extrainfo{additional information}
% \photo[70pt][0.4pt]{pictures/picture} % The first bracket is the picture height, the second is the thickness of the frame around the picture (0pt for no frame)
% \quote{"My basic interest in clustering is not to produce algorithms, but rather because I think that classification is necessary as a foundation to probability." - John Hartigan}

%----------------------------------------------------------------------------------------

\begin{document}

%----------------------------------------------------------------------------------------
%	COVER LETTER
%----------------------------------------------------------------------------------------

% To remove the cover letter, comment out this entire block

%\clearpage
%
%\recipient{HR Department}{Corporation\\123 Pleasant Lane\\12345 City, State} % Letter recipient
%\date{\today} % Letter date
%\opening{Dear Sir or Madam,} % Opening greeting
%\closing{Sincerely yours,} % Closing phrase
%\enclosure[Attached]{curriculum vit\ae{}} % List of enclosed documents
%
%\makelettertitle % Print letter title
%
%\lipsum[1-2] % Dummy text
%\lipsum[4] % Dummy text
%
%\makeletterclosing % Print letter signature
%
%\newpage

%----------------------------------------------------------------------------------------
%	CURRICULUM VITAE
%----------------------------------------------------------------------------------------

\makecvtitle % Print the CV title

%----------------------------------------------------------------------------------------
%	EDUCATION SECTION
%----------------------------------------------------------------------------------------

\section{Education  \hfill \normalsize GPA: 3.8 Overall, 4.0 In-Major}
\cventry{2018}{Masters of Science in Computer Science}{Wright State University}{Dayton, OH}{}{} 
\cventry{2015}{Bachelor of Science in Computer Science}{Wright State University}{Dayton, OH}{\textit{Minor in Statistics}}{}

\subsection{Relevant Courses Taken}
\begin{center}
\begin{tabularx}{\linewidth}{ X X X }
 $\bullet$ Network Science &  $\bullet$ Machine Learning & $\bullet$ Information Theory  \\ 
 $\bullet$ Applied Stochastic Processes & $\bullet$ Applied Statistics I \& II  & $\bullet$ Algorithm Design and Analysis  \\
  $\bullet$ Empirical Analysis & $\bullet$ Optimization Techniques & $\bullet$ Foundations of AI  \\  
 $\bullet$ Computational Tools and \; \mbox{Techniques} for Data Analysis &  $\bullet$ Theoretical Statistics & $\bullet$ Advanced Programming Languages

\end{tabularx}
\end{center}

%----------------------------------------------------------------------------------------
%	SUBSECTION: MASTERS THESIS 
%----------------------------------------------------------------------------------------

%\section{Masters Thesis \hfill }
%
%\cvitem{Title}{\emph{Money Is The Root Of All Evil -- Or Is It?}}
%\cvitem{Supervisors}{Professor James Smith \& Associate Professor Jane Smith}
%\cvitem{Description}{This thesis explored the idea that money has been the cause of untold anguish and suffering in the world. I found that it has, in fact, not.}

%----------------------------------------------------------------------------------------
%	WORK EXPERIENCE SECTION
%----------------------------------------------------------------------------------------

\section{Research Experience}
\textbf{Research Interests: } My research interests are in unsupervised learning, statistical learning theory, computational geometry, and building software for the purpose of scientific computing and reproducible research.  
\newline 

%------------------------------------------------
\cventry{Oak Ridge Institute for Science and Education}
			 {Air Force Research Laboratory}
			 {Research Associate}
			 {Fall 2018-Present, Fall 2017}{}{\vspace{3pt}
			 In a collaborative effort to foster new research frontiers in the area of Topology Data Analysis (TDA) between WSU and AFRL, I worked in a research group studying how to combine techniques from  the field of topology and machine learning in both supervised and unsupervised settings. I primarily researched theoretical extensions to the \textit{Mapper} framework, an often used modality for performing TDA. 
			 My work lead to the development of an algorithmic solution which greatly reduces the complexity of the \textit{Mapper} framework and enables a more tractable analysis of {\em mapper} constructions in the context of Persistent Homology. A journal article demonstrating the utility of this solution is currently in development, but available as a draft, see~\cite{mapperext}. 
}

%------------------------------------------------
\cventry{National Aeronautics and Space Administration }
			 {John H. Glenn Research Center at Lewis Field}
			 {LERCIP Intern}
			 {Summer 2018}{}{\vspace{3pt}
I was hired by Dr. Steven Arnold in the Multiscale Modeling Materials and Structures Division for a 10-week internship at NASA to use Machine Learning to explore the possibility automatically capturing process-property-structure (PSP) relationships through the use of a ML-based surrogate model trained on data generated from the Generalized Method of Cells.
An additional requirement of the project (per NASAs Vision 2040 guidelines) was to incorporate experimental design methodology for interpreting the results.  
The project involved:
\begin{itemize}
	\item Learning basic mechanics and lamination theory
	\item Architecting a feed-forward neural network (the surrogate model) to model laminate stress-response data 
	\item Modifying an optimization procedure (approximate coordinate exchange) to minimize a different loss function (conditional mutual information) to generate optimal designs
\end{itemize}	
A technical report and subsequent journal is planned for the future. Presentation material, code, and a draft of the technical report is available upon request for U.S. citizens only. 
}

%------------------------------------------------

\cventry{Web and Complex Systems Lab}
			 {Wright State University}
			 {Graduate Research Assistant}
			 {2015 - 2018}{}
{  \vspace{3pt}
While at the WaCS lab, the topic areas I've focused on include: 
	\begin{itemize}
		\item Density-based clustering techniques and theory 
		\item Dynamic or Temporal Network Models
		\item Trajectory mining and modeling
	\end{itemize}
The research project I was assigned aimed to modeling real-world traffic networks at a macroscopic scale. The goal of the project is to turn raw positioning/track information into a dynamic network representation, and then model that representation. 
Much of my work in the project involved researching viable theory-first approaches to large-scale density-based clustering. Specifically, my work has focused on augmenting the cluster tree, a shape characteristic of an estimated density function, with semi-supervised information for purpose of point of interest (POI) discovery in geospatial contexts~\cite{poi_paper}, to be used as the `vertices' in the dynamic network representation. 
My research is supported by the Center for Surveillance Research, a National Science Foundation I/UCRC.
}

%------------------------------------------------

\cventry{R Project for Statistical Computing / Google}
			 {Google Summer of Code 2017}
			 {Student Participant}
			 {Summer 2017}{}{\vspace{3pt} % Begin summary 
			 I submitted a successful funding proposal under the Google Summer of Code (GSOC) Initiative to the R Project for Statistical Computing to explore, develop, and unify developments related the theory of density-based clustering, namely the recent developments related to the cluster tree. This involved a mixture of code development which culminated in the form of an R package, as well as deep research to further understand the theory and utility of the cluster tree. For more details, see the project page\footnotemark. 
			 } 
\footnotetext{\url{https://summerofcode.withgoogle.com/archive/2017/projects/5919718795902976/}}
%------------------------------------------------

\cventry{Oak Ridge Institute for Science and Education}
			 {Air Force Institute of Technology}
			 {Student Research Associate}
			 {2014 - 2016}{}{\vspace{3pt}
I worked on the development of a novel Iterative Closest Point algorithm amenable to massive parallelization, implemented in C++/CUDA, for the purposes of enabling real-time tracking of aircraft in the context of Autonomous Aerial Refueling. The effort lead to multiple publications~\cite{piekenbrock2016automated, robinson2016parallelized}. I also worked on:
\begin{itemize}
	\item Parallelizing existing atmospheric absorption routines with OpenCL through MATLABs MEX interface
	\item A model for predicting web navigation patterns using Hierarchical Markov Models
	\item A prototypical UI to to enhance searching and viewing of 3D models using ThreeJS
\end{itemize}
}

%------------------------------------------------
\cventry{Southwestern Ohio Council for Higher Education}
			 {Air Force Institute of Technology}
			 {Undergraduate Research Assistant}
			 {2013 - 2014}{}{\vspace{3pt}
As my first part-time position in academia, I worked on a diverse set of projects, often assisting graduate or doctoral students working in the research area with primarily programmatic or educational tasks. This involved: 
\begin{itemize}
	\item Codifying a novel nonlinear optimization algorithm in ANSI-C
	\item Implementing an unsplittable flow approximation algorithm in C++ and Python
	\item Creating a conversion tool that allowed for converting back and forth between Oracle's Abstract Data Type specification to its equivalent representation as an XMLType 
\end{itemize}	
}
%----------------------------------------------------------------------------------------
%	PUBLICATIONS SECTION
%----------------------------------------------------------------------------------------
\section{Publications}

%% Refereed Publications
% \subsection{\Large Refereed Publications}
%% Stuff under development 
\nocitedev{poi_paper, mapperext}
\bibliographystyledev{plainyr-rev}
\bibliographydev{my_bib.bib}

%% Journals 
\nocitejournal{hahslerdbscan}
\bibliographystylejournal{plainyr-rev}
\bibliographyjournal{my_bib.bib}

%% Conference Papers 
% \nociteproceeding{robinson2016parallelized} 
\nociteproceeding{piekenbrock2016automated}
\nociteproceeding{robinson2017seasonality}
\nociteproceeding{maurice2015waminet}
\nociteproceeding{robinson2016parallelized}
\bibliographystyleproceeding{plainyr-rev}
\bibliographyproceeding{my_bib.bib}

%% Abstracts 
\nociteabstract{sunbelt}
\bibliographystyleabstract{plainyr}
\bibliographyabstract{my_bib.bib}


%----------------------------------------------------------------------------------------
%	PACKAGES SECTION
%----------------------------------------------------------------------------------------
\section{Open Source Contributions}
{\bfseries dbscan (R package)}\footnote{\url{https://github.com/mhahsler/dbscan}} \hfill Coauthor \\
{\bfseries clustertree (R package)}\footnote{\url{https://github.com/peekxc/clustertree}} \hfill Author \\ 
{\bfseries Mapper (R package)}\footnote{\url{https://github.com/peekxc/mapper}} \hfill Author 
%\cvitem{DBSCAN R package}{\hfill Coauthor}
%\cvitem{clustertree R package}{\hfill Author}
%\cvitem{Mapper R package}{\hfill Author}


%----------------------------------------------------------------------------------------
%	AWARDS SECTION
%----------------------------------------------------------------------------------------

\section{Awards, Extra Curricular, Misc.}

%\cvitem{2017}{Google Summer of Code - Funding }
\cvitem{Outstanding Masters Student Award (Computer Science)}{\hfill WSU 2017-2018 academic year}
\cvitem{Student participant and presenter}{\hfill NSF TRIPODS TGDA Summer School and Workshop}
\cvitem{Regional Model United Nations Annual Conference}{\hfill Served in Volunteer Staff  (2016 - 2017)}
\cvitem{Outstanding Position Paper Award}{\hfill National Model United Nations Annual Conference (2014)}
\cvitem{Outstanding Delegation Award}{\hfill National Model United Nations Annual Conference (2013)}

%----------------------------------------------------------------------------------------
%	COMPUTER SKILLS SECTION
%----------------------------------------------------------------------------------------

%\section{Computer skills}
%
%\cvitem{Basic}{\textsc{java}, Adobe Illustrator}
%\cvitem{Intermediate}{\textsc{python}, \textsc{html}, \LaTeX, OpenOffice, Linux, Microsoft Windows}
%\cvitem{Advanced}{Computer Hardware and Support}

%----------------------------------------------------------------------------------------
%	COMMUNICATION SKILLS SECTION
%----------------------------------------------------------------------------------------

%\section{Communication Skills}
%
%\cvitem{2010}{Oral Presentation at the California Business Conference}
%\cvitem{2009}{Poster at the Annual Business Conference in Oregon}

%----------------------------------------------------------------------------------------
%	LANGUAGES SECTION
%----------------------------------------------------------------------------------------

%\section{Languages}
%
%\cvitemwithcomment{English}{Mothertongue}{}
%\cvitemwithcomment{Spanish}{Intermediate}{Conversationally fluent}
%\cvitemwithcomment{Dutch}{Basic}{Basic words and phrases only}

%----------------------------------------------------------------------------------------
%	Awards SECTION
%----------------------------------------------------------------------------------------

%\section{Interests}
%
%\renewcommand{\listitemsymbol}{-~} % Changes the symbol used for lists
%
%\cvlistdoubleitem{Piano}{Chess}
%\cvlistdoubleitem{Cooking}{Dancing}
%\cvlistitem{Running}

%----------------------------------------------------------------------------------------
%	INTERESTS SECTION
%----------------------------------------------------------------------------------------

%\section{Interests}
%
%\renewcommand{\listitemsymbol}{-~} % Changes the symbol used for lists
%
%\cvlistdoubleitem{Piano}{Chess}
%\cvlistdoubleitem{Cooking}{Dancing}
%\cvlistitem{Running}

%----------------------------------------------------------------------------------------

\end{document}